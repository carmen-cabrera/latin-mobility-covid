\RequirePackage[l2tabu,orthodox]{nag}
\documentclass[11pt,letterpaper]{article}
\usepackage[T1]{fontenc}
\usepackage[utf8]{inputenc}
\usepackage{crimson}
\usepackage{helvet}
\usepackage[strict,autostyle]{csquotes}
\usepackage[USenglish]{babel}
\usepackage{microtype}
\usepackage{authblk}
\usepackage{booktabs}
\usepackage{caption}
\usepackage{endnotes}
\usepackage{geometry}
\usepackage{graphicx}
\usepackage{hyperref}
\usepackage{natbib}
\usepackage{rotating}
\usepackage{setspace}
\usepackage{titlesec}
\usepackage{url}
\usepackage{hyperref}
\usepackage{soul}
\usepackage[dvipsnames]{xcolor}
\usepackage[many]{tcolorbox}
\newtcolorbox{mybox}{colback = black!5!gray!50, colframe = black!75!black, segmentation style={solid}} % create text box
\usepackage{hanging}

% location of figure files, via graphicx package
\graphicspath{{./figures/}}

% configure the page layout, via geometry package
\geometry{
	paper=letterpaper,
	top=2.5cm,
	bottom=2.5cm,
	left=2.5cm,
	right=2.5cm}
\setstretch{1.02}
\clubpenalty=10000
\widowpenalty=10000

% set section/subsection headings as the sans serif font
\titleformat{\section}{\normalfont\sffamily\large\bfseries}{\thesection.}{0.3em}{}
\titleformat{\subsection}{\normalfont\sffamily\small\bfseries}{\thesubsection.}{0.3em}{}

% make figure/table captions sans-serif small font
\captionsetup{font={footnotesize,sf},labelfont=bf,labelsep=period}

% configure pdf metadata and link handling
\hypersetup{
	pdfauthor={Carmen Cabrera-Arnau},
	pdftitle={The mid-term impact of COVID-19 on human mobility patterns in Latin American countries},
	pdfsubject={Title},
	pdfkeywords={Keywords},
	pdffitwindow=true,
	breaklinks=true,
	colorlinks=false,
	pdfborder={0 0 0}}

\title{The mid-term impact of COVID-19 on human mobility patterns in Latin American countries\footnote{\textbf{Citation}: Cabrera-Arnau, C., González-Leonardo, M., Nasuto, A., Neville, R., Rowe, F. (2023). The mid-term impact of COVID-19 on human mobility patterns in Latin American countries}}
\author[1]{Carmen Cabrera-Arnau \thanks{\textit{Corresponding author}: c.cabrera-arnau@liverpool.ac.uk}}
\author[2]{Miguel González-Leonardo}
\author[1]{Andrea Nasuto}
\author[1]{Ruth Neville}
\author[1]{Francisco Rowe}
\affil[1]{Geographic Data Science Lab, Department of Geography and Planning, University of Liverpool, Liverpool, United Kingdom}
\affil[2]{Center for Demographic, Urban and Environmental Studies, El Colegio de México (COLMEX), Mexico City, Mexico}

\date{}

% From pandoc:
% https://github.com/jgm/pandoc-templates/blob/master/default.latex
\setlength{\emergencystretch}{3em} % prevent overfull lines
\providecommand{\tightlist}{%
  \setlength{\itemsep}{0pt}\setlength{\parskip}{0pt}}

% https://stackoverflow.com/questions/41052687/rstudio-pdf-knit-fails-with-environment-shaded-undefined-error


\begin{document}

\maketitle


\begin{abstract}

Recent empirical studies, predominantly from countries in the Global North, have shown that the COVID-19 pandemic disrupted human mobility at different spatial scales, but particularly in big cities, initially identified as epicentres of infections. Lockdowns, remote work, and online education decreased the demand for commuting and urban living, resulting in an "urban exodus". Despite the existing evidence, little is known about whether counterurbanisation movements have unfolded similarly in the Global South. In this study, we use anonymised, high-resolution location data from Facebook users to answer the following research questions: i) to what extent are people leaving cities? ii) are there variations across countries? iii) are there changes in mobility patterns over time? iv) how long do these changes last? The data is aggregated into spatial units according to the Bing Maps Tile System and temporally aggregated into 8-hour time windows for a period of approximately 2 years. The methodology involves three steps. Firstly, we classify the Bing tiles into different population density categories according to population data from WorldPop. Secondly, we quantify and analyse the number and characteristics of movements between different locations in each country. Finally, we use statistical modelling and spatial interaction models to assess how the population density category at the origin and destination affects the strength of the flows at different points in time. We observe that the strength of flows between Bing tiles belonging to different population density categories was systematically reduced in the early days of the pandemic and, even though by 2022 it seemed to have increased, it never returned to pre-pandemic levels. In the early days of COVID-19, we also find a systematically increase above pre-pandemic levels of the strength of the flows between Bing tiles belonding to the the same population density category. As of 2022, the frequency of these movements within population density classes have decreased with respect to 2020.

%\vspace{1cm}
\end{abstract}



\pagebreak






% print the bibliography
\setlength{\bibsep}{0.00cm plus 0.05cm} % no space between items
\bibliographystyle{apalike}
\bibliography{sim_refs}



\end{document}
