% Options for packages loaded elsewhere
\PassOptionsToPackage{unicode}{hyperref}
\PassOptionsToPackage{hyphens}{url}
\PassOptionsToPackage{dvipsnames,svgnames,x11names}{xcolor}
%
\documentclass[
  11pt,
]{article}

\usepackage{amsmath,amssymb}
\usepackage{lmodern}
\usepackage{iftex}
\ifPDFTeX
  \usepackage[T1]{fontenc}
  \usepackage[utf8]{inputenc}
  \usepackage{textcomp} % provide euro and other symbols
\else % if luatex or xetex
  \usepackage{unicode-math}
  \defaultfontfeatures{Scale=MatchLowercase}
  \defaultfontfeatures[\rmfamily]{Ligatures=TeX,Scale=1}
\fi
% Use upquote if available, for straight quotes in verbatim environments
\IfFileExists{upquote.sty}{\usepackage{upquote}}{}
\IfFileExists{microtype.sty}{% use microtype if available
  \usepackage[]{microtype}
  \UseMicrotypeSet[protrusion]{basicmath} % disable protrusion for tt fonts
}{}
\makeatletter
\@ifundefined{KOMAClassName}{% if non-KOMA class
  \IfFileExists{parskip.sty}{%
    \usepackage{parskip}
  }{% else
    \setlength{\parindent}{0pt}
    \setlength{\parskip}{6pt plus 2pt minus 1pt}}
}{% if KOMA class
  \KOMAoptions{parskip=half}}
\makeatother
\usepackage{xcolor}
\usepackage[left=2.54cm,right=2.54cm]{geometry}
\setlength{\emergencystretch}{3em} % prevent overfull lines
\setcounter{secnumdepth}{5}
% Make \paragraph and \subparagraph free-standing
\ifx\paragraph\undefined\else
  \let\oldparagraph\paragraph
  \renewcommand{\paragraph}[1]{\oldparagraph{#1}\mbox{}}
\fi
\ifx\subparagraph\undefined\else
  \let\oldsubparagraph\subparagraph
  \renewcommand{\subparagraph}[1]{\oldsubparagraph{#1}\mbox{}}
\fi


\providecommand{\tightlist}{%
  \setlength{\itemsep}{0pt}\setlength{\parskip}{0pt}}\usepackage{longtable,booktabs,array}
\usepackage{calc} % for calculating minipage widths
% Correct order of tables after \paragraph or \subparagraph
\usepackage{etoolbox}
\makeatletter
\patchcmd\longtable{\par}{\if@noskipsec\mbox{}\fi\par}{}{}
\makeatother
% Allow footnotes in longtable head/foot
\IfFileExists{footnotehyper.sty}{\usepackage{footnotehyper}}{\usepackage{footnote}}
\makesavenoteenv{longtable}
\usepackage{graphicx}
\makeatletter
\def\maxwidth{\ifdim\Gin@nat@width>\linewidth\linewidth\else\Gin@nat@width\fi}
\def\maxheight{\ifdim\Gin@nat@height>\textheight\textheight\else\Gin@nat@height\fi}
\makeatother
% Scale images if necessary, so that they will not overflow the page
% margins by default, and it is still possible to overwrite the defaults
% using explicit options in \includegraphics[width, height, ...]{}
\setkeys{Gin}{width=\maxwidth,height=\maxheight,keepaspectratio}
% Set default figure placement to htbp
\makeatletter
\def\fps@figure{htbp}
\makeatother

\usepackage[noblocks]{authblk}
\renewcommand*{\Authsep}{, }
\renewcommand*{\Authand}{, }
\renewcommand*{\Authands}{, }
\renewcommand\Affilfont{\small}
\usepackage{lipsum} \usepackage{libertine}
\makeatletter
\makeatother
\makeatletter
\makeatother
\makeatletter
\@ifpackageloaded{caption}{}{\usepackage{caption}}
\AtBeginDocument{%
\ifdefined\contentsname
  \renewcommand*\contentsname{Table of contents}
\else
  \newcommand\contentsname{Table of contents}
\fi
\ifdefined\listfigurename
  \renewcommand*\listfigurename{List of Figures}
\else
  \newcommand\listfigurename{List of Figures}
\fi
\ifdefined\listtablename
  \renewcommand*\listtablename{List of Tables}
\else
  \newcommand\listtablename{List of Tables}
\fi
\ifdefined\figurename
  \renewcommand*\figurename{Figure}
\else
  \newcommand\figurename{Figure}
\fi
\ifdefined\tablename
  \renewcommand*\tablename{Table}
\else
  \newcommand\tablename{Table}
\fi
}
\@ifpackageloaded{float}{}{\usepackage{float}}
\floatstyle{ruled}
\@ifundefined{c@chapter}{\newfloat{codelisting}{h}{lop}}{\newfloat{codelisting}{h}{lop}[chapter]}
\floatname{codelisting}{Listing}
\newcommand*\listoflistings{\listof{codelisting}{List of Listings}}
\makeatother
\makeatletter
\@ifpackageloaded{caption}{}{\usepackage{caption}}
\@ifpackageloaded{subcaption}{}{\usepackage{subcaption}}
\makeatother
\makeatletter
\@ifpackageloaded{tcolorbox}{}{\usepackage[many]{tcolorbox}}
\makeatother
\makeatletter
\@ifundefined{shadecolor}{\definecolor{shadecolor}{rgb}{.97, .97, .97}}
\makeatother
\makeatletter
\makeatother
\ifLuaTeX
  \usepackage{selnolig}  % disable illegal ligatures
\fi
\IfFileExists{bookmark.sty}{\usepackage{bookmark}}{\usepackage{hyperref}}
\IfFileExists{xurl.sty}{\usepackage{xurl}}{} % add URL line breaks if available
\urlstyle{same} % disable monospaced font for URLs
\hypersetup{
  pdftitle={Urban mobility in Latin America after COVID-19},
  pdfauthor={Carmen Cabrera-Arnau; Francisco Rowe; Miguel González-Leonardo; Andrea Nasuto; Ruth Neville},
  colorlinks=true,
  linkcolor={blue},
  filecolor={Maroon},
  citecolor={Blue},
  urlcolor={Blue},
  pdfcreator={LaTeX via pandoc}}

\title{\textbf{Urban mobility in Latin America after COVID-19}}


\author[1]{Carmen Cabrera-Arnau}
\author[1]{Francisco Rowe}
\author[2]{Miguel González-Leonardo}
\author[1]{Andrea Nasuto}
\author[1]{Ruth Neville}

\affil[1]{Geographic Data Science Lab, Department of Geography and
Planning, University of Liverpool, Liverpool, UK}
\affil[2]{Centre for Demographic Urban and Environmental Studies, El
Colegio de México, Ciudad de México, México}


\date{}
\begin{document}
\maketitle
\begin{abstract}
The COVID‐19 pandemic has impacted the national systems of population
movement around the world.
\end{abstract}
\ifdefined\Shaded\renewenvironment{Shaded}{\begin{tcolorbox}[enhanced, interior hidden, sharp corners, boxrule=0pt, borderline west={3pt}{0pt}{shadecolor}, frame hidden, breakable]}{\end{tcolorbox}}\fi

\hypertarget{sec-intro}{%
\section{Introduction}\label{sec-intro}}

Spatial human mobility is key to creating sustainable, liveable and
inclusive cities. At the societal level, spatial mobility enables the
transfer of knowledge, skills and labour to places they are needed.
Spatial mobility also shapes service and transport demand across urban
spaces, and enables the monitoring and control of transmissible
diseases. At the individual level, spatial mobility enables people to
access and achieve opportunities and aspirations. Understanding spatial
human mobility is thus important to supporting appropriate policy
responses to address societal challenges relating to carbon emissions,
urban planning, service delivery, public health, disaster management and
transport.

The COVID-19 pandemic resulted in a notable reduction in mobility within
cities. Nonpharmaceutical interventions to contain the spread of
COVID‐19 reduced overall levels of mobility. Especially during
lockdowns, mobility recorded reductions in the frequency, distance and
time of trips in cities across the globe. Rises in teleworking, online
schooling and remote shopping activity reduced the need to travel for
work, education, shopping and leisure. Coupled with fears of crowded
public spaces, nonpharmaceutical interventions prompted more
geographically localised mobility patterns.

However, reductions in mobility levels were highly unequal reflecting
existing socioeconomic inequalities. In most countries, affluent
individuals tended to record the greatest drops in mobility levels as
they are predominantly employed in knowledge-intensive jobs which can be
done fully or partly remotely. During the COVID-19 pandemic, the
adoption of remote work reduced the need of commuting for
knowledge-intensive, non-public facing jobs. At the same time,
individuals from less privileged socioeconomic backgrounds displayed
less pronounced declines mirroring the nature of their work requiring
public-facing, face-to-face interaction, and thus a requirement for
daily work commutes.

Thus, while a growing body of empirical evidence has contributed to
advancing our understanding of the impacts of the COVID-19 pandemic on
spatial mobility within cities, existing research has focused on more
developed countries and the immediate effects of the pandemic during
2020. Less is known about the longer term patterns of resilience in
urban mobility in less developed countries. Assessing the extent to
which the intensity of movement has returned back to pre-pandemic levels
across socioeconomic groups is important to understand the potentially
unequal impacts of hybrid working in the society.

A key barrier to continously monitor changes in geographic mobility
patterns in less developed countries during and post the COVID-19
pandemic has been the lack of suitable data. Traditionally census and
survey data have been employed to analyse human mobility patterns in
these countries. Yet, these data streams are not frequently updated and
suffer from slow releases, with census data for example being collected
over intervals of ten years in most countries. These data streams thus
lack the temporal granularity to analyse population movements over
short-time periods. Data resulting from social interactions on digital
platforms has emerged as an unique source of information to capture
human population movement at scale in less developed countries.
Particularly location data from mobile phone applications have become a
prominent source to sense patterns of human mobility at higher
geographical and temporal resolution over in real time.

We aim to

\begin{itemize}
\tightlist
\item
  Context and importance of human mobility: public health , climate
  change\\
\item
  The pandemic resulted in major changes in human mobility - unequal
  impacts across socioeconomic groups\\
\item
  Gap 1: Little work on understanding the long-term changes in mobility
  - recovery\\
\item
  Gap 2: little work on less developed countries - where inequalities
  are more pronounced\\
\item
  Aim - focus on South American countries as case study.
\end{itemize}

\hypertarget{sec-results}{%
\section{Results}\label{sec-results}}

The evolution of the levels of movement was measured with respect to a
baseline period prior to the pandemic as described in
Section~\ref{sec-methods}. For the purposes of the analysis, we
aggregate the raw movement data temporally into months and spatially
into administrative units according to various GADM levels. The analysis
focuses on administrative areas that are within the boundaries of
functional urban areas as specified by the Global Human Settlement
Layer. For each administrative area, we compute the Relative Deprivation
Index based on data from NASA's Socioeconomic Data and Applications
Centre (SEDAC). Figure X displays the administrative areas included in
the study, coloured according to the Relative Index of Deprivation.
Predictions about the evolution of the levels of movement are made using
the Prophet forecasting procedure. Further details are provided in
Section~\ref{sec-methods}.

\hypertarget{the-impact-of-covid-19-on-urban-mobility}{%
\subsection{The impact of COVID-19 on urban
mobility}\label{the-impact-of-covid-19-on-urban-mobility}}

We analysed the evolution of the relative intensity of urban mobility
with respect to a baseline period prior to the pandemic. Specifically,
movements covering a distance of at most 70 km are considered. For a
movement to be classified as urban it needs to start or end within a
functional urban area from Argentina, Chile, Colombia and Mexico. The
observed data is available for a two-year period starting in April 2020,
just after the first wave of COVID-19 pandemic cases, and ending in
March 2022. After 2022 no observations are available, however, we
generate a 12-month forecast up to March 2023 in order to gain a better
understanding of the recovery trends.

Figure X displays the patterns of recovery for the mobility levels in
the administrative units belonging to functional urban areas in the
countries of interest. The three lines in each panel represent the mean
levels of mobility for administrative units grouped into one of three
terciles, according to their average relative deprivation index.

Generally, there was a drop in the levels of mobility with respect to
the baseline period in all four countries. This drop was especially
large for Argentina, Chile and Colombia, with Mexico displaying a
smaller decrease in the number of movements with respect to the
baseline. Following the initial drop in movement, all four countries
evolve towards the recovery of baseline levels of urban mobility, with a
generally increasing trend. There are fluctuations from the general
trend, which manifest differently for each country. These fluctuations
mirror each other in the case of Argentina and Colombia, where urban
mobility sharply bounces back closer to pre-pandemic levels around July
of 2020. Chile and Mexico display more progressive patterns of recovery,
although Chile never reaches baseline levels. These fluctuations are
unique to each country and can be attributed to local factors such as
the effects of seasonality or the different stringency measures imposed
by the national governments during the pandemic.

From Figure X, we observe that there is a consistent tendency in how
administrative units with varying levels of deprivation were affected by
the pandemic. For all four countries, we observe that the administrative
units in the most deprived tercile are the ones that experienced the
smallest loss in levels of mobility at the beginning of the pandemic.
Differences in the levels of mobility across relative deprivation
terciles diminish with time. Argentina and Chile stand out as the
countries with the largest differences in mobility levels for different
relative deprivation terciles.

\hypertarget{socioeconomic-deprivation-and-recovery-of-urban-mobility}{%
\subsection{Socioeconomic deprivation and recovery of urban
mobility}\label{socioeconomic-deprivation-and-recovery-of-urban-mobility}}

In this section we explore further the role of socioeconomic deprivation
in the evolution of the levels of urban mobility. For a given point in
time (i.e.~a month), we start by considering the relationship between
the number of movements relative to the pre-pandemic baseline period and
the average relative deprivation index, at the administrative unit
level. We assume that this relationship is linear and we use a linear
regression to estimate the slope and intercept characterising the line
of best fit. This is shown for April 2020 and March 2022 in the
right-hand side panels of Figure X. After obtaining the slope and
intercept for every month, we are able to plot the evolution of these
parameters for both the observed and forecasted data, as displayed on
the left-hand-side panels of the same Figure.

We find patterns in the evolution of the estimated parameters that
characterise the relationship between the levels of urban mobility and
RDI. In Argentina, Colombia and Mexico, we observe that the slope of
this relationship evolves to become smaller over time. The tendency is
not apparent in Chile, where the slope of the relationship remains
approximately the same despite the temporary fluctuations. The slope
captures the extent of differences in the level of urban mobility across
administrative units with varying levels of socioeconomic deprivation.
It can therefore be regarded as a measure of inequality. A slope equal
to zero would mean that all administrative units display the same
intensity of movement regardless of their socioeconomic deprivation
levels. Given the patterns observed in Argentina, Colombia and Mexico,
we find that at the beginning of the pandemic there were notable
inequalities between socioeconomic groups in terms of the levels of
urban mobility. While it has taken more than two years for Argentina and
Mexico to close the gap (their slope is close to zero from spring 2022),
inequalities persist in Chile and Colombia as of March 2023.

The intercept of the relationship displays stronger patterns, which are
consistent across the four countries. The intercept captures the urban
mobility levels in administrative where relative deprivation is zero, or
in other words, if there was no socioeconomic deprivation in an
administrative area, the intercept would represent its level of
mobility. The intercept was well below the baseline level at the early
stages of the pandemic. As observed in Figure X, while there are some
differences between countries in the evolution of the intercept, the
general tendency is for the intercept to increase. While Argentina and
Mexico reach values that are closer to the baseline towards the end of
the forecast period, the intercept for Chile and Colombia remains lower.
Therefore, if there were areas with no socioeconomic deprivation, we
would have seen a recovery in the levels of mobility, although not quite
back to baseline levels in the case of Chile and Colombia.

\hypertarget{sec-discussion}{%
\section{Discussion}\label{sec-discussion}}

\hypertarget{sec-methods}{%
\section{Methods}\label{sec-methods}}

\hypertarget{meta-facebook-movement-data}{%
\subsection{Meta-Facebook movement
data}\label{meta-facebook-movement-data}}

To capture population movements, we used anonymised aggregate mobile
phone location data from Meta users for Argentina, Colombia, Chile and
Mexico, covering a 24-month period from April 2020 to March 2022. We
used the dataset Facebook Movements created by Meta and accessed through
their Data for Good Initiative (https://dataforgood.facebook.com). The
data are built from Facebook app users who have the location services
setting turned on on their smartphone. Prior to releasing the datasets,
Meta ensures privacy and anonymity by removing personal information and
applying privacy-preserving techniques (Maas et al.~2019). Small-count
dropping is one of these techniques. A data entry is removed if the
population or movement count for an area is lower than 10. The removal
of small counts may mean that population counts in small sparsely
populated areas are not captured. A second technique consists in adding
a small undisclosed amount of random noise to ensure that it is not
possible to ascertain precise, true counts for sparsely populated
locations. Third, spatial smoothing using inverse distance-weighted
averaging is also applied is applied to produce a smooth population
count surface. The Facebook Movements dataset offers information on the
total number of Facebook users moving between and within spatial units
in the form of origin-destination matrices. The data is temporally
aggregated into three daily 8-hour time windows (i.e.~00:00-08:00,
08:00-16:00 and 16:00-00:00). The dataset includes a baseline capturing
the number of movements before COVID-19 based on a 45-day period ending
on March 10th 2020. The baseline is computed using an average for the
same time of the day and day of the week in the period preceding March
10th. For instance, the baseline for Monday 00:00-08:00 time window is
obtained by averaging over data collected on Mondays from 00:00 to 8:00
for the 45-day period. Details about the baseline can be found in Maas
et al.~(2019). The Bing Maps Tile System developed by Microsoft
(Microsoft) is used a spatial framework to organise the data. The Tile
System is a geospatial indexing system that partitions the world into
tile cells in a hierarchical way, comprising 23 different levels of
detail (Microsoft). At the lowest level of detail (Level 1), the world
is divided into four tiles with a coarse spatial resolution. At each
successive level, the resolution increases by a factor of two. The data
that we used are spatially aggregated into Bing tile levels 13. That is
about 4.9 x 4.9km at the Equator (Maas et al.~2019).

\hypertarget{spatiotemporal-data-aggregation}{%
\subsection{Spatiotemporal data
aggregation}\label{spatiotemporal-data-aggregation}}

Since the focus of this work is on urban mobility, we focus the analysis
on Functional Urban Areas (FUAs), defined by the Global Human Settlement
Layer. The spatial extent of the FUAs is often large and may include
several towns and neighbourhoods displaying a variety of socioeconomic
characteristics, hence using FUAs as the spatial units of aggregation
would considerably mask the heterogeneity in the mobility patterns.
While the original unit of aggregation for the movement data, i.e.~the
tiles from the Bing Maps Tile System, offers the highest degree of
spatial granularity available, the interpretation of findings from an
analysis based on administrative units is often more valuable. For this
reason, we perform a spatial join to aggregate the movement data into
administrative units at the GADM level 2 or 3. Only flows of people
starting or ending within the boundaries of a FUA are considered.

While the original movement data is originally aggregated into 8-hour
windows, this resolution is too fine for our analysis. Since the
analysis is focused on the longer-term evolution of patterns of urban
mobility, we aggregate the movement data by month.

In our analysis, we use the Relative Deprivation Index (RDI) as a
measure of socioeocnomic deprivation. The RDI data is made available via
NASA's Socioeconomic Data and Applications Centre (SEDAC), with a
spatial resolution of 1km pixels. We perform a spatial join of the
gridded data and the administrative units and compute the average RDI
within each of these units.

\hypertarget{time-series-analysis}{%
\subsection{Time series analysis}\label{time-series-analysis}}

For the countries included in the analysis, the movement data is
available until March 2022. In order to gain a better understanding of
the recovery trends after the pandemic, we generate a 12-month forecast
up to March 2023. This is done using Prophet, a procedure to forecast
time series data based on an additive model where non-linear trends can
be fit with seasonality effects, holiday effects and other external
factors such as the stringency index. For the analysis, only seasonality
effects are included as they yield the most realistic predictions.

In Figures X and Y, outliers are removed. Outliers are defined as
administrative units which, at any given month, have a z-score greater
than 4 for the relative intensity of movement. Furthermore, a
Savitzky-Golay filter is applied to smooth the time series data, using a
length of 4 units for the filter window and order 2 polynomials to fit
the samples.

\hypertarget{references}{%
\section{References}\label{references}}



\end{document}
