\RequirePackage[l2tabu,orthodox]{nag}
\documentclass[11pt,letterpaper]{article}
\usepackage[T1]{fontenc}
\usepackage[utf8]{inputenc}
\usepackage{crimson}
\usepackage{helvet}
\usepackage[strict,autostyle]{csquotes}
\usepackage[USenglish]{babel}
\usepackage{microtype}
\usepackage{authblk}
\usepackage{booktabs}
\usepackage{caption}
\usepackage{endnotes}
\usepackage{geometry}
\usepackage{graphicx}
\usepackage{hyperref}
\usepackage{natbib}
\usepackage{rotating}
\usepackage{setspace}
\usepackage{titlesec}
\usepackage{url}
\usepackage{hyperref}
\usepackage{soul}
\usepackage[dvipsnames]{xcolor}
\usepackage[many]{tcolorbox}
\newtcolorbox{mybox}{colback = black!5!gray!50, colframe = black!75!black, segmentation style={solid}} % create text box
\usepackage{hanging}

% location of figure files, via graphicx package
\graphicspath{{./figures/}}

% configure the page layout, via geometry package
\geometry{
	paper=letterpaper,
	top=2.5cm,
	bottom=2.5cm,
	left=2.5cm,
	right=2.5cm}
\setstretch{1.02}
\clubpenalty=10000
\widowpenalty=10000

% set section/subsection headings as the sans serif font
\titleformat{\section}{\normalfont\sffamily\large\bfseries}{\thesection.}{0.3em}{}
\titleformat{\subsection}{\normalfont\sffamily\small\bfseries}{\thesubsection.}{0.3em}{}

% make figure/table captions sans-serif small font
\captionsetup{font={footnotesize,sf},labelfont=bf,labelsep=period}

% configure pdf metadata and link handling
\hypersetup{
	pdfauthor={Carmen Cabrera-Arnau},
	pdftitle={The influence of socioeconomic deprivation on the recovery of urban mobility after COVID-19 in Latin America},
	pdfsubject={Title},
	pdfkeywords={Keywords},
	pdffitwindow=true,
	breaklinks=true,
	colorlinks=false,
	pdfborder={0 0 0}}

\title{The influence of socioeconomic deprivation on the recovery of urban mobility after COVID-19 in Latin America}
\author[1]{Carmen Cabrera-Arnau \thanks{\textit{Corresponding author}: c.cabrera-arnau@liverpool.ac.uk}}
\author[2]{Miguel González-Leonardo}
\author[1]{Andrea Nasuto}
\author[1]{Ruth Neville}
\author[1]{Francisco Rowe}
\affil[1]{Geographic Data Science Lab, Department of Geography and Planning, University of Liverpool, Liverpool, United Kingdom}
\affil[2]{Center for Demographic, Urban and Environmental Studies, El Colegio de México (COLMEX), Mexico City, Mexico}

\date{}

% From pandoc:
% https://github.com/jgm/pandoc-templates/blob/master/default.latex
\setlength{\emergencystretch}{3em} % prevent overfull lines
\providecommand{\tightlist}{%
  \setlength{\itemsep}{0pt}\setlength{\parskip}{0pt}}

% https://stackoverflow.com/questions/41052687/rstudio-pdf-knit-fails-with-environment-shaded-undefined-error


\begin{document}

\maketitle


\begin{abstract}



%\vspace{1cm}
\end{abstract}



\pagebreak

\section{Introduction}

\subsection{General context}

\subsection{New forms of digital data to track human mobility}

\subsection{Contribution}

\section{Data}

The aim of this study is to uncover the role of socioeconomic
deprivation in determining the pace of post-pandemic recovery of urban
mobility in Latin America. To do this, we leverage high-resolution data
on human mobility collected by Facebook and on relative deprivation
collected by .

\subsection{Facebook-Meta data}

\subsection{Relative deprivation data}

\subsection{Other data}

Our analysis focuses on intraurban mobility. The analysis is done at two
spatial scales: for functional urban areas (FUAs) and for smaller
spatial units: communes.

The rationale for this is: FUAs give us an indication of commuter
regions, so they are the natural spatial unit to track intra-urban
mobility. However, there is a lot of socioeconomic inequalities in Latin
America, so we want to know what happens at smaller spatial scales, such
as the commune level. For this reason, we also use data from W, X, Y and
Z, which give us the geographical delimitations for FUAs in the four
countries analysed here respectively, as well as delimitations for the
communes forming the different FUAs.

\section{Methods}

\subsection{Determining recovery}

Explain here the baseline level of mobility and how just after the
pandemic, there was a generalised drop in mobility across the
urban-rural continuum. As the pandemic evolved and some of the
stringency measures were relaxed, the levels of mobility bounced back to
levels closer to those observed during the baseline period. Show Figure
illustrating this evolution.

However, for the subsequent parts of the analysis, we need a formal
definition of ``recovery'' since we want to quantitatively assess the
association between relative deprivation and the pace of return to
normal levels of urban mobility.

For each spatial unit (FUA or commune) the recovery time \(t_R^*\) is
defined as the time at which we can reject the hypothesis (\(p<0.05\))
that the level of mobility \(m(t)\) is below the baseline level. The
level of mobility \(m\) could be the percentage change relative to the
baseline of the number of trips per user in a given spatial unit.

To do this, we need to apply the following method:

\begin{itemize}

\item Model the time evolution of $m$ taking into account exogenous regressors such as the specific month of the year (to account for seasonality?), the average population density and, importantly, the stringency index. The model needs to be predictive and needs to give the probability distribution of the outcome variable.
\item For the model, each sample will be formed by a spatial unit. Each sample has as many time steps as months in the movement dataset. Each sample also has information that should be incorporated as exogenous regressors.
\item Compute $t_R^*$ such that $ t_R = min\{t_R: Pr(m(t_R)<0) < 0.05\}$.
\end{itemize}

\subsection{Analysis}

For each spatial unit, we will be left with a value of \(t_R*\). We can
also compute the average relative deprivation of each spatial unit,
denoted by \(d\). Then, we can check whether there is an association
between these two variables. Do we observe a trend wherer cities that
reach recovery first are those where deprivaton is lower? If so, which
are those cities?

\section{Results}

\section{Discussion}




% print the bibliography
\setlength{\bibsep}{0.00cm plus 0.05cm} % no space between items
\bibliographystyle{apalike}
\bibliography{refs}



\end{document}
